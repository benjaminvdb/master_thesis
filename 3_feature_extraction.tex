\chapter{Feature Extraction}
\label{ch:feature_extraction}

In the previous chapter we described how to obtain references from documents and how confidences scores are computed that reflect how certain we are that they refer to the same entity.
During the link classification step we assumed that records consisted only of information directly relating to a reference, e.g. first names, last names, roles, titles, etc.
If the probability mass function of values for a property is known it can be incorporated, but it does not extend the records themselves.
When we consider the case of a reference containing a \emph{common name}, however, it becomes apparent that more information is required.
The confidence score is inversely proportional to the frequency of the reference's components, yielding a low score in the case of a common name.
This can merely improve the fraction of correctly classified matches (precision), but it cannot avoid missing matches because of a low score (recall).
In order to achieve that, more information is required.
Looking at the documents from which references are extracted, we see that more data available that could potentially be useful in aiding classification, is available.

In this chapter we will look at two complementary approaches that extract contextual information.
The first is based on modeling the content of a section of text in which a reference appears in, while the second is based on co-occurrence of other people.
The goal is to extract information from the text such that ambiguous cases, such as ``John Smith'', can still be linked.


%\subsection{Rule-based reference extraction}
% I am discussing this as the first step of the pipeline now.

%\subsection{NLP-driven reference extraction (?)}
% I will probably not discuss this topic in detail, though I might use it to compare the described rule-based system to.


% % % % % % % % % % % % % % % % % % % % % %


\section{$k$-Maximally informative itemsets}
\label{sec:miki}

The primary target for the methods studied in this thesis are historical documents.
Often these documents comprise the proceedings of a meeting or a summarized version thereof.
They contain a description of events that took place on a specific date, who was present, etc.
When looking more closely at the contents it is often possible to extract a few interesting words.
As an example, consider the following sections taken from ``Officials of the Boards of Trade and Plantations''\footnote{http://www.british-history.ac.uk/office-holders/vol3}.

\begin{quote}
    A letter from the Secretary to Mr. Carkesse, desiring him to move the Commissioners of the Customs, that their Officers in the Out Ports may give this Board an Account of the quantities of Salt that is necessary and used in curing several species of Fish, was agreed and ordered to be sent.
\end{quote}

\begin{quote}
    Ordered that Mr. Carkesse be desired to let this Board have on Tuesday next, if possible, the Account of Fish exported, which was desired the 17th of the last month.
\end{quote}

\noindent In both cases there is a reference to a ``Mr. Carkesse'', which might be the same person.
Both texts talk discuss matter that is related to the export of fish, so we could say the overarching topic is ``fish''.
The fact that Mr. Carkesse is mentioned twice in the context of this topic gives us additional information that we can exploit to determine whether these references should be matched.
Note that if there would have been multiple people called Mr. Carkesse, probably additional information would have been provided.
It is therefore unlikely that two different persons are referred to.
However, it is a good example of the kind of ``circumstantial evidence'' that we want to extract from the documents.

Any word from the text may potentially be used as a topic for the text; in fact, we could select all of them.
However, this would lead to topics, treated as features of the text, to be highly correlated.
If we were to select ``fish'' as a topic and decide to also include ``fishing'', it is easy to see that probably not much additional information is included.
Besides trying to decide on a topic by looking at a single document, we can instead look at the complete set of document under consideration and construct a set of uncorrelated, or \emph{orthogonal}, topics.
This is very much related to the process of \emph{features selection}, that strives to select a subset of features with the purpose of reducing problems caused when too many features are used, such as overfitting.

In our case we aim for the selection of a set of topics that provide as good a distinction between texts as possible.
This is exactly what \emph{maximally informative $k$-itemsets}\cite{knobbe06}, or \emph{miki's} in short, provide.
As the name suggests, a miki is an itemset of size $k$ that provides maximum information.
The amount of information of the itemset is measured as the \emph{joint entropy} of the itemset in \emph{Shannon bits} (or just \emph{bits}), which is defined as follows: \cite{knobbe06}

\begin{definition}[Joint entropy]
    Suppose that $X = \left\{ x_{1}, \dots, x_{k} \right\}$ is an itemset, and $B = \left( b_{1}, \dots, b_{k} \right) \in \left\{ 0, 1 \right\}^{k}$ is a tuple of binary values. The \emph{joint entropy} of $X$ is defined as
    \begin{equation*}
        H(X) = -\sum_{B \in \left\{ 0, 1 \right\}^{k}} p \left( x_{1} = b_{1}, \dots, x_{k} = b_{k} \right) \lg p \left( x_{1} = b_{1}, \dots, x_{k} = b_{k} \right)
    \end{equation*}
    \label{def:joint_entropy}
\end{definition}

\begin{definition}[Maximally informative $k$-itemset]
    Suppose that I is a collection of $n$ items. An itemset $X \subseteq I$ of cardinality $k$ is a \emph{maximally informative k-itemset}, $iff$ for all itemsets $Y \subseteq I$ of cardinality $k$,
    \begin{equation}
        H(Y) \leq H(X)
    \end{equation}
    \label{def:miki}
\end{definition}

We will write a $k$-element subset of $I$ as a list of integers that refer to the elements of $I$.

\begin{equation*}
    A = \left[ x_{1}, \dots, x_{k} \right]
\end{equation*}

\noindent where

\begin{equation*}
    x_{1} < \dots < x_{k}
\end{equation*}

Entropy can also be explained a measure of disorder or uncertainty, i.e., the higher the entropy of an item the more unpredictable it is.
For example, a toss of a fair coin is maximally unpredictable and has an entropy of $1$, since its two outcomes ``heads'' and ``tails'' are equally likely.

Analogous to the coin toss example, an itemset $I$ of maximum entropy consists of elements that are uniformly distributed over the data, i.e., a sample contains an item or combination of items of $I$ with equal probability.
This is best illustrated with an example.

\begin{table}
\centering
    \begin{minipage}{.3\textwidth}
        \centering
        \begin{tabular}{?c ^c ^c ^c}
            \toprule
            \rowstyle{\bfseries}
            A & B & C & D \\
            \midrule
            1 & 1 & 1 & 0 \\
            1 & 1 & 0 & 0 \\
            1 & 1 & 1 & 0 \\ 
            1 & 0 & 0 & 0 \\
            0 & 1 & 1 & 0 \\
            0 & 0 & 0 & 1 \\
            0 & 0 & 1 & 1 \\
            0 & 0 & 0 & 1 \\
            \bottomrule
        \end{tabular}
    \end{minipage}%
    \begin{minipage}{.3\textwidth}
        \centering
        \sisetup{round-mode=places}
        \begin{tabular}{c S[round-precision=3]}
            \toprule
            $I$ & $H$ \\
            \midrule
            A & 1.0 \\
            B & 1.0 \\
            C & 1.0 \\
            D & 0.95443400292496494 \\
            \bottomrule
        \end{tabular}
    \end{minipage}%
    \begin{minipage}{.3\textwidth}
        \centering
        \sisetup{round-mode=places}
        \begin{tabular}{c c S[round-precision=3]}
            \toprule
            $I_{1}$ & $I_{2}$ & $H$ \\
            \midrule
            A & B & 1.8112781244591329 \\
            A & C & 2.0 \\
            A & D & 1.4056390622295665 \\
            B & B & 1.8112781244591329 \\
            B & D & 1.4056390622295665 \\
            C & D & 1.9056390622295665 \\
            \bottomrule
        \end{tabular}
    \end{minipage}
\caption{An example dataset of size $8$ and cardinality $4$.}
\label{tab:joint_entropy}
\end{table}

\cref{tab:joint_entropy} shows a small example dataset of size $8$ and cardinality $4$. Items $A$, $B$ and $C$ are uniformly distributed over the dataset, so they are $1$-miki's.
They are however very similar, since there is a lot of overlap, so they are redundant.
When we sample the dataset and find that it contains $A$ it is quite likely that it will also contain $B$.
Knowing whether the sample contains $A$ or not gives us a lot of information about whether or not $B$ is included.
The additional information provided by $B$ is therefore low, so its joint entropy will be relatively low.


The dataset is partitioned in such a way that the distribution of its parts are as uniform as possible.

\begin{itemize}
    \item We may try to model the various contexts using topics and provide the record linker with that information, represented as a binary vector.
    \item This means that we need a way of extracting topics from text.
    \item Documents are sequences of words and each of them has associated to a probability distribution of the words used.
    \item Preferably mutual exclusive partitions of equal size.
    \item Entropy is the expected value (average) of the information contained in each message received. Here, message stands for an event, sample or character drawn from a distribution or data stream. Entropy thus characterizes our uncertainty about our source of information, and increases for more sources of greater randomness
    \item The logarithm of the probability distribution is useful as a measure of entropy because it is additive for independent sources. For instance, the entropy of a coin toss is 1 shannon, whereas of m tosses it is m shannons. Generally, you need log2(n) bits to represent a variable that can take one of n values if n is a power of 2. If these values are equiprobable, the entropy (in shannons) is equal to the number of bits.
    \item $H$
\end{itemize}


% % % % % % % % % % % % % % % % % % % % % %


\section{Co-occurrence (?)}
\label{sec:co-occurrence}

\begin{quote}
    Ordered that a letter be writ to Mr. Carkesse, Secretary to the Commrs. of his Majesty's Customs, to desire Mr. John Burgoyn, Deputy Registrar General of all trading ships may attend this Board any morning except Saturdays and Mondays.
\end{quote}

\begin{quote}
    Mr. Burgoyn, Deputy Register General of all trading ships attending, their lordships had some discourse with him, and afterwards ordered that a letter be writ to Mr. Carkesse to move the Commrs. of his Majesty's Customs for directions to Mr. Burgoyn to lay before this Board, as soon as conveniently may be, an account of the number of ships, that have cleared from England from Christmas, 1709 to Christmas, 1714 specifying from what ports they have cleared, and to what ports they went, and an account of the tunnage of the said ships, distinguishing the British from the foreign ships.
\end{quote}