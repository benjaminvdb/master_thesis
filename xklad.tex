\documentclass{article}

\usepackage{mystyle}

\newcommand{\desc}{\bm\delta}  % Description symbol (bold delta)
\newcommand{\descx}[1]{\bm{\delta}_\mathrm{#1}}  % Named description
\newcommand{\entity}{\bm{e}}  % Entity vector
\newcommand{\cardin}[1]{\left\vert #1 \right\vert}  % Cardinality
\newcommand{\stringify}[1]{\ensuremath{\text{``#1''}}}
\newcommand{\describes}{\ensuremath{\approx_\mathrm{desc}}}


%\overset{\left[#1\right]}{\underset{\mathrm{H}}{=}}}

\begin{document}

In order to define what we mean with entity resolution, let us first define what we mean with entities and how their descriptions are defined.
Since it is hard to define a domain that will be suited for any entity, we will use references to entities in an unknown domain $E$.
Assume we can obtain information about entities through an `oracle' function $Q$, defined on an entity $\bm{e} \in E$ and a property key $k \in \Sigma^{*}$ that maps to a proper value if the entity possesses the property and a `missing value' $\epsilon$ otherwise.
Using alphabetic characters as keys, we can for example obtain the number of corners of a cube: $Q(\descx{cube}, \stringify{corners}) = 8$.

A \emph{description} $\desc$ consists of a tuple of keys $K$ and values $V$,
e.g. $\descx{cube}=((\stringify{corners}, \stringify{sides}), (8, 6))$.
We denote with the $\cardin{\desc}$ the \emph{description length}, defined as $\cardin{K}=\cardin{V}$, thus $\cardin{\descx{cube}}=2$ in the previous example.
A description $\desc = (K, V)$ is said to describe an entity $\entity$, denoted with $\desc \describes \entity$, whenever the properties of the description match that of the entity, i.e., $Q(k_i, \entity) = v_i$ with $i = 1, 2, \dots, \cardin{\desc}$.
To allow for small mistakes and variation in the data, we will describe more precisely how to compare properties in \cref{sec:field_comparisons}.

We can now describe the process of entity resolution as follows:

\begin{definition}[Entity resolution]
    Entity resolution is the process of deriving from a set of descriptions $\Delta$ another set of descriptions $\hat{\Delta}$ s.t. each of the described entities $\entity \in E$ are described by \emph{at most one description} (injective, non-surjective) and \emph{no information is lost}, i.e.,
    \begin{equation*}
        \hat{\Delta} = \left\{ \bm{x} \describes \entity \mid \bm{x} \describes \entity \wedge \bm{y} \describes \entity \implies \bm{x} = \bm{y}, \forall \bm{x}, \bm{y} \right\}
    \end{equation*}
\end{definition}

Note that even in the presence of an oracle $Q$, we could never know if a description truly references an entity, since the non-surjective property implies that a description can describe several elements in $E$.
What makes matters even more complicated is that we also lack knowledge of the set $E$.
The best we can do is assume that descriptions describe at least one entity in $E$ and `ground' the oracle to it, i.e., we treat descriptions as incomplete entities.
This allows us to do pairwise comparisons of descriptions as the core component of the entity resolution process.
Whenever a query involves an unknown property, an $\epsilon$ is returned.

Also note that from the fact that there is an injective, non-surjective \emph{describes}-relation between $\hat{\Delta}$ and $E$ and that no information is lost, it is implied that descriptions in the original set $\Delta$ are \emph{merged} during the record linkage process.
Entity resolution therefore focuses on the construction of a set of unique descriptions, which involves the aggregation of information whenever two partial descriptions exist for the same (real-world) entity.
Record linkage, however, is only concerned with the detection of the latter and not with the aggregation itself.

\begin{definition}[Record linkage]
    Given a set of (partial) descriptions $\Delta$, record linkage is the process of determining the set of \emph{matching pairs} $M$, or simply \emph{matches}, defined as:
    \begin{equation*}
        M = \left\{ (\bm{x}, \bm{y}) \mid \bm{x} \describes \entity \wedge \bm{y} \describes \entity, \exists \entity \in E, \forall \bm{x}, \bm{y} \in \Delta \right\}
    \end{equation*}
    \label{def:record_linkage}
\end{definition}

Again, by grounding the oracle to a specific description, we can do pairwise comparisons between descriptions.
Note that the relation defined in \cref{def:record_linkage} is \emph{symmetric}, which limits the total number of possible matches to $n(n-1)/2$, if we do not compare descriptions with themselves.

\end{document}