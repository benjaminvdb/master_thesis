\documentclass{article}

\begin{document}

\begin{itemize}
    \item \cite{Winchester1970} page 4 explains the use of documenting the steps that lead to the linkage of records and gives an example of a project.
    \item \cite{Winchester1970} studies record linkage from the perspective of a historian. It is noted that though the record linkage task seems trivial, in practice it is often hard to execute manually. In the same paper this exemplified with a few truly linked records pairs: Yelverton, J.H.P.G. (turner)/Silvester, J.H.P.G. (sawyer); Horan, Michael (laborer)/Hograth, Michael (laborer); Mayplant, William (carman)/Murfitt, William (teamster); Clagris, Philip (chairmaker)/Claykes, Philip (chairmaker).
    \item "Wikify! Linking Documents to Encyclopedic Knowledge" includes a mention of the application of entity resolution to creating a semantic web.
    \item In natural language processing, entity linking, named entity disambiguation (NED), named entity recognition and disambiguation (NERD) or named entity normalization (NEN) (van: https://en.wikipedia.org/wiki/Entity_linking)
    \item Replace "person references" to "names entities" or something more formal and specific.
    \item Replace "empty value" to "null value" for clarity.
    \item Use dot notation to refer to a field of an entity $E$, e.g. $E.first_name$.
    \item Investigate "Likert scale"
    \item In 1998, Merrill Lynch cited a rule of thumb that somewhere around 80-90 of all potentially usable business information may originate in unstructured form.[1] This rule of thumb is not based on primary or any quantitative research, but nonetheless is accepted by some. (Wikipedia)
    \item Use ``acronym'' or ``glossaries'' package to create a list of acronyms which can be linked to from the text.
    \item Create an index for the thesis using the ``index'' package. ``xindy'' seems a good candidate and can be used with the ``glossaries'' package.
    \item It is important to note that even though we can similarity metrics to perform ``fuzzy'' matching, name variants can still cause problems and it is better to learn these first and deal with them.
    \item Perhaps it's interesting to do some quick evaluation of approximate MIKis versus true MIKIs. This could be done by cross-validating on smaller datasets derived from e.g. Wikipedia using random sampling.
    \item Transitivity can lead to inconsistencies in the inference of the record linker. However, evidence can be obtained by applying transitivity, since nodes that share a relatively high number of edges, i.e., are graphically similar, might refer to the same entity.
    \item Though it is not the primary goal of this thesis, it might be interesting to consider an example of a possible next step in the process. Discuss how an entity-driven web search might aggregate information about a person.
    \item 
\end{itemize}

\bibliographystyle{plain}
\bibliography{mybib}

\end{document}