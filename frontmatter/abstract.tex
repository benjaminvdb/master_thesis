Archives around the world store innumerable amounts of historical data.
In the last few years, many of these archives have been digitized, which enables new ways of exploring the data.
Computer applications can search through hundreds of years of data in just seconds.
However, many of these datasets lack a structure that is easily interpretable by computers.
In order to be able to construct the comprehensive life stories of individuals that are hidden within these datasets, the references to individuals should be extracted.
Furthermore, it should be decided which of these references refer to the same real-world entities in a process called \emph{entity resolution}.

This thesis discusses the methods and techniques that can be used for this purpose.
A software pipeline is described that takes unstructured text as input and outputs reference pairs, ranked by a confidence score.
This score is based on the rareness of the properties of the reference, such as name and occupation, and information extracted from the surrounding context.
The novelty of this research was the introduction of the \emph{Maximally k-Informative Itemset} (MIKI) as a means of capturing the topics of a section that a reference occurs in.
Experiments were conducted on real archive data supplied by the The National Archives in London.
For a portion of this data, entity resolution had been carried out manually by historians.
This allowed us to compare theirs results with that of our automated technique.

We found that even entity resolution can be performed automatically in many cases, though our technique has difficulties in linking references of which little information is known.
Better methods are therefore required that are able to extract more information from the context in which references appear.